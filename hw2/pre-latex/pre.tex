\documentclass[10pt, journal]{IEEEtran}

\hyphenation{op-tical net-works semi-conduc-tor}

\usepackage[T1]{fontenc} 
\usepackage{amsmath}
\usepackage[cmintegrals]{newtxmath}
\usepackage{bm}



\begin{document}
\title{Engineering Electromagnetics - Experiment 2\\ Electromagnetics of Line Charge}
\author{\IEEEauthorblockN{QingFu~Qin}, 
\IEEEauthorblockA{Southern University of Science and Technology, ShengZhen, GuangDong}\\
\IEEEauthorblockA{Email: 11910103@mail.sustech.edu.cn}
        
% \thanks{Mr.Jia was with the Department
% of Electrical and Computer Engineering, Georgia Institute of Technology, Atlanta,
% GA, 30332 USA e-mail: (see http://www.michaelshell.org/contact.html).}
}

\maketitle


\begin{abstract}
    This article describes the electric feild of the line charge which lenth is 2.
    By using MATLAB to simulate the electric feild and draw the picture of 
    analysis the difference between two  
\end{abstract}

\end{document}